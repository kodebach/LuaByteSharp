\documentclass{article}

\usepackage[margin=2cm]{geometry}
\usepackage[utf8]{luainputenc}

\usepackage{hyperref}
\usepackage{longtable}
\usepackage{tabu}
\usepackage{multicol}
\usepackage{sectsty}

\sectionfont{\fontsize{24}{28}\selectfont}
\subsectionfont{\fontsize{16}{18}\selectfont}
\subsubsectionfont{\fontsize{14}{16}\selectfont}

\usepackage{lmodern}
\renewcommand*\familydefault{\sfdefault}
\usepackage[T1]{fontenc}

\let\oldparagraph\paragraph
\renewcommand{\paragraph}[1]{\oldparagraph{#1}\mbox{}\\}

\begin{document}
\section*{LuaByteSharp}\hypertarget{luabytesharp}{}\label{luabytesharp}

\subsection*{Kurzbeschreibung des Projekts}\hypertarget{kurzbeschreibung-des-projekts}{}\label{kurzbeschreibung-des-projekts}

LuaByteSharp soll ein Interpreter für Lua Byte-Code werden. Der Interpreter soll in C\# geschrieben werden.\newline
Das Programm soll fähig sein, den vom Lua Standardcompiler luac, in der Version 5.3, erzeugten Byte-Code einzulesen und auszuführen.\newline
Geplant ist, dass alle der 47 existierenden Operationen (\href{https://www.lua.org/source/5.3/lopcodes.h.html}{siehe}) unterstützt werden.\newline
Als Quellen für die Beschreibung der Opertationen, sowie des generellen Formats eines LuaByteCode Files, werden die unter \hyperlink{quellen}{Quellen} gelisteten Quellen verwendet.
Geplant ist ebenfalls, dass das Metatable-Konzept (\href{https://www.lua.org/manual/5.3/manual.html\#2.4}{siehe}) von Lua unterstützt wird.

Da Lua-Code oft auf der Lua Standard Library aufbaut, soll zumindest ein Subset der Standard Library unterstützt werden.\newline
Das Subset wird wahrscheinlich je nach verfügbarer Zeit angepasst werden. Jedenfalls werden aber weder \texttt{require} noch \texttt{load} bzw. \texttt{loadfile} und infolgedessen auch nicht \texttt{dofile} untersützt werden, da alle diese Funktionen Lua Sourcecode laden können, der Interpreter diesen aber nicht parsen kann.

Außerdem muss das Lua Byte Code File kompatibel zur 64-Bit Version von luac sein, und im Little-Endian Format vorliegen. Es muss folgender Header vorhanden sein:

\begin{verbatim}00:    1B 4C 75 61 53 00 19 93
08:    0D 0A 1A 0A 04 08 04 08
10:    08 78 56 00 00 00 00 00
18:    00 00 00 00 00 00 28 77
20:    40
\end{verbatim}

\subsection*{Detailierte Auflistung der Bestandteile}\hypertarget{detailierte-auflistung-der-bestandteile}{}\label{detailierte-auflistung-der-bestandteile}

\subsubsection*{Byte Code Instruktionen}\hypertarget{byte-code-instruktionen}{}\label{byte-code-instruktionen}

siehe \href{https://www.lua.org/source/5.3/lopcodes.h.html}{hier}\newline
bzw. \href{https://github.com/dibyendumajumdar/ravi/blob/master/readthedocs/lua\_bytecode\_reference.rst}{hier}

\subsubsection*{Standard Library Funktionen}\hypertarget{standard-library-funktionen}{}\label{standard-library-funktionen}

\href{https://www.lua.org/manual/5.3/manual.html\#6}{siehe auch}

\paragraph{Basic Functions}\hypertarget{basic-functions}{}\label{basic-functions}

Bis auf die folgenende Ausnahmen, sollen alle Basic Functions unterstützt werden:

\begin{multicols}{2}
\begin{itemize}
\item \texttt{collectgarbage}
\item \texttt{dofile}
\item \texttt{load}
\item \texttt{loadfile}
\item \texttt{pcall}
\item \texttt{xpcall}
\end{itemize}
\end{multicols}

\pagebreak

\paragraph{String Manipulation}\hypertarget{string-manipulation}{}\label{string-manipulation}

Auch hier ist geplant, bis auf einige Ausnahmen, alle Funktionen zu unterstützen.

Ausnahmen:

\begin{multicols}{2}
\begin{itemize}
\item \texttt{string.dump}
\item \texttt{string.pack}
\item \texttt{string.packsize}
\item \texttt{string.unpack}
\end{itemize}
\end{multicols}

\paragraph{UTF-8 Support}\hypertarget{utf-8-support}{}\label{utf-8-support}

Die folgenden Methoden sollen alle unterstützt werden:

\begin{multicols}{2}
\begin{itemize}
\item \texttt{utf8.char}
\item \texttt{utf8.charpattern}
\item \texttt{utf8.codes}
\item \texttt{utf8.codepoint}
\item \texttt{utf8.len}
\item \texttt{utf8.offset}
\end{itemize}
\end{multicols}

\paragraph{Table Manipulation}\hypertarget{table-manipulation}{}\label{table-manipulation}

Die folgenden Methoden sollen alle unterstützt werden:

\begin{multicols}{2}
\begin{itemize}
\item \texttt{table.concat}
\item \texttt{table.insert}
\item \texttt{table.move}
\item \texttt{table.pack}
\item \texttt{table.remove}
\item \texttt{table.sort}
\item \texttt{table.unpack}
\end{itemize}
\end{multicols}

\paragraph{Mathematical Functions}\hypertarget{mathematical-functions}{}\label{mathematical-functions}

Alle mathematischen Funktionen der Lua Standard Library sollen unterstützt werden.

\subsubsection*{Mögliche Erweiterungen}\hypertarget{mögliche-erweiterungen}{}\label{mögliche-erweiterungen}

Möglicherweise werden auch alle oder Teile der IO Funktionen unterstützt werden.\newline
Auch die Funktionen \texttt{string.pack}, \texttt{string.packsize} und \texttt{string.unpack} werden vielleicht unterstützt.

\subsection*{Aufwandsabschätzungen}\hypertarget{aufwandsabschätzungen}{}\label{aufwandsabschätzungen}

\begin{longtabu}{|X[l]|l|}
\hline
\bfseries Komponente & \bfseries Aufwand\\
\hline
Interpreter Grundgerüst & 10 h\\
Byte Code Instruktionen & 20 h\\
Standard Library & 40 h\\
\hline
\end{longtabu}

\pagebreak

\subsubsection*{Byte Code Instruktionen}\hypertarget{byte-code-instruktionen-1}{}\label{byte-code-instruktionen-1}

\begin{longtabu} to \textwidth {|l|X[l]|l|}
\hline
\bfseries Gruppe & \bfseries Instruktionen & \bfseries Aufwand\\
\hline
Move/Load & \texttt{MOVE}, \texttt{LOADK}, \texttt{LOADKX}, \texttt{LOADBOOL}, \texttt{LOADNIL} & 2 h\\
Up-Value/Table & \texttt{(GET/SET)UPVAL}, \texttt{(GET/SET)TABUP}, \texttt{(GET/SET)TABLE}, \texttt{NEWTABLE} & 2 h\\
Arithmetik & \texttt{ADD}, \texttt{SUB}, \texttt{MUL}, \texttt{DIV}, \texttt{IDIV}, \texttt{MOD}, \texttt{POW}, \texttt{UNM} & 1 h\\
Bit Manipulation & \texttt{BAND}, \texttt{BOR}, \texttt{BXOR}, \texttt{SHL}, \texttt{SHR}, \texttt{BNOT} & 1 h\\
Logik & \texttt{TEST}, \texttt{TESTSET}, \texttt{NOT} & 1 h\\
Vergleiche & \texttt{EQ}, \texttt{LT}, \texttt{LE} & 1 h\\
Schleifen & \texttt{FORLOOP}, \texttt{FORPREP}, \texttt{TFORLOOP}, \texttt{TFORCALL} & 2 h\\
Funktionen & \texttt{CALL}, \texttt{TAILCALL}, \texttt{RETURN}, \texttt{CLOSURE} & 6 h\\
Andere & \texttt{JMP}, \texttt{VARARG}, \texttt{SETLIST}, \texttt{CONCAT}, \texttt{LEN}, \texttt{SELF}, \texttt{EXTRAARG} & 4 h\\
\hline
\end{longtabu}

\subsubsection*{Standard Library Funktionen}\hypertarget{standard-library-funktionen-1}{}\label{standard-library-funktionen-1}

\begin{longtabu} to \textwidth {|l|X[l]|l|}
\hline
\bfseries Gruppe & \bfseries Funktionen & \bfseries Aufwand\\
\hline
Basis Funktionen & \texttt{assert}, \texttt{error}, \texttt{\_G}, \texttt{getmetatable}, \texttt{ipairs}, \texttt{next}, \texttt{pairs}, \texttt{print}, \texttt{rawequal}, \texttt{rawget}, \texttt{rawlen}, \texttt{rawset}, \texttt{select}, \texttt{tonumber}, \texttt{tostring}, \texttt{type}, \texttt{\_VERSION} & 10 h\\
String Manipulation & \texttt{string.byte}, \texttt{string.char}, \texttt{string.find}, \texttt{string.format}, \texttt{string.gmatch}, \texttt{string.gsub}, \texttt{string.len}, \texttt{string.lower}, \texttt{string. match}, \texttt{string.rep}, \texttt{string.reverse}, \texttt{string.sub}, \texttt{string.upper} & 15 h\\
Mathematik & \texttt{math.abs}, \texttt{math.acos}, \texttt{math.asin}, \texttt{math.atan}, \texttt{math.ceil}, \texttt{math.cos}, \texttt{math.deg}, \texttt{math.exp}, \texttt{math.floor}, \texttt{math.fmod}, \texttt{math.huge}, \texttt{math.log}, \texttt{math.max}, \texttt{math.maxinteger}, \texttt{math.min}, \texttt{math.mininteger}, \texttt{math.modf}, \texttt{math.pi}, \texttt{math.rad}, \texttt{math.random}, \texttt{math.randomseed}, \texttt{math.sin}, \texttt{math.sqrt}, \texttt{math.tan}, \texttt{math.tointeger}, \texttt{math.type}, \texttt{math.ult} & 5 h\\
Table Manipulation & \texttt{table.concat}, \texttt{table.insert}, \texttt{table.move}, \texttt{table.pack}, \texttt{table.remove}, \texttt{table.pack}, \texttt{table.remove}, \texttt{table.sort}, \texttt{table.unpack} & 5 h\\
UTF-8 Support & \texttt{utf8.char}, \texttt{utf8.charpattern}, \texttt{utf8.codes}, \texttt{utf8.codepoint}, \texttt{utf8.len}, \texttt{utf8.offset} & 5 h\\
\hline
\end{longtabu}

\newpage
\subsection*{Quellen}\hypertarget{quellen}{}\label{quellen}

\subsubsection*{Lua Byte Code File Format}\hypertarget{lua-byte-code-file-format}{}\label{lua-byte-code-file-format}

\begin{itemize}
\item http://luaforge.net/docman/83/98/ANoFrillsIntroToLua51VMInstructions.pdf (nur teilweise)
\item http://files.catwell.info/misc/mirror/lua-5.2-bytecode-vm-dirk-laurie/lua52vm.html (nur teilweise)
\item https://www.lua.org/source/5.3/ldump.h.html
\item https://www.lua.org/source/5.3/ldump.c.html
\item https://www.lua.org/source/5.3/lundump.h.html
\item https://www.lua.org/source/5.3/lundump.c.html
\end{itemize}

\subsubsection*{Byte Code Instruktionen}\hypertarget{byte-code-instruktionen-2}{}\label{byte-code-instruktionen-2}

\begin{itemize}
\item https://www.lua.org/source/5.3/lopcodes.h.html
\item https://www.lua.org/source/5.3/lvm.c.html
\item https://github.com/dibyendumajumdar/ravi/blob/master/readthedocs/lua\_bytecode\_reference.rst
\end{itemize}

\subsubsection*{Standard Library Funktionen}\hypertarget{standard-library-funktionen-2}{}\label{standard-library-funktionen-2}

\begin{itemize}
\item https://www.lua.org/manual/5.3/manual.html\#6
\item https://www.lua.org/source/5.3/lbaselib.c.html
\item https://www.lua.org/source/5.3/lstrlib.c.html
\end{itemize}

\end{document}